\section*{Rules of Inference} \label{rulesofinference}

\begin{axiom_appendix}[Modus Ponens]
$(P \wedge (P \implies Q)) \implies Q $
\end{axiom_appendix}

\begin{axiom_appendix}[Modus Tollens]
$(\neg Q \wedge (P \implies Q)) \implies \neg P $
\end{axiom_appendix}

\begin{axiom_appendix}[Hypothetical Syllogism (transitivity)]
\hfill

$((P \implies Q) \wedge (Q \implies R)) \implies (P \implies R) $
\end{axiom_appendix}

\begin{axiom_appendix}[Disjunctive Syllogism]
$((P \vee Q) \wedge \neg P) \implies Q$
\end{axiom_appendix}

\begin{axiom_appendix}[Addition]
$P \implies (P \vee Q)$
\end{axiom_appendix}

\begin{axiom_appendix}[Simplification]
$(P \wedge Q) \implies P$
\end{axiom_appendix}

\begin{axiom_appendix}[Conjunction]
$((P) \wedge (Q)) \implies (P \wedge Q)$
\end{axiom_appendix}

\begin{axiom_appendix}[Resolution]
$((P \vee Q) \wedge (\neg P \vee R)) \implies (Q \vee R)$
\end{axiom_appendix}

\section*{Laws of Logic}

\begin{axiom_appendix}[Implication Law]
$(P \implies Q) \equiv (\neg P \vee Q)$
\end{axiom_appendix}

\begin{axiom_appendix}[Distributive Law]
\begin{align*}
    (P \wedge (Q \vee R)) &\equiv ((P \wedge Q) \vee (P \wedge R)) \\
    (P \vee (Q \wedge R)) &\equiv ((P \vee Q) \wedge (P \vee R))
\end{align*}
\end{axiom_appendix}

\begin{axiom_appendix}[De Morgan's Law]
\begin{align*}
    \neg (P \wedge Q) &\equiv (\neg P \vee \neg Q) \\
    \neg (P \vee Q) &\equiv (\neg P \wedge \neg Q)
\end{align*}
\end{axiom_appendix}

\begin{axiom_appendix}[Absorption Law]
\begin{align*}
    (P \vee (P \wedge Q)) &\equiv P \\
    (P \wedge (P \vee Q)) &\equiv P
\end{align*}
\end{axiom_appendix}

\begin{axiom_appendix}[Commutativity of AND]
$A \wedge B \equiv B \wedge A$
\end{axiom_appendix}

\begin{axiom_appendix}[Associativity of AND]
$(A \wedge B) \wedge C \equiv A \wedge (B \wedge C)$
\end{axiom_appendix}

\begin{axiom_appendix}[Identity of AND]
$\textbf{T} \wedge A \equiv A$
\end{axiom_appendix}

\begin{axiom_appendix}[Zero of AND]
$\textbf{F} \wedge A \equiv \textbf{F}$
\end{axiom_appendix}

\begin{axiom_appendix}[Idempotence for AND]
$A \wedge A \equiv A$
\end{axiom_appendix}

\begin{axiom_appendix}[Contradiction for AND]
$A \wedge \neg A \equiv \textbf{F}$
\end{axiom_appendix}

\begin{axiom_appendix}[Double Negation]
$\neg (\neg A) \equiv A$
\end{axiom_appendix}

\begin{axiom_appendix}[Validity for OR]
$A \vee \neg A \equiv \textbf{T}$
\end{axiom_appendix}

\section*{Induction}

\vspace{\parskip}

\begin{axiom_appendix}[Well Ordering Principle]
Every nonempty set of nonnegative integers has a smallest element. i.e., For any $A \subset \N$ such that $A \neq \emptyset$, there is some $a \in A$ such that $\forall a' \in A.a\leq a'$.
\end{axiom_appendix}

\section*{Recurrences}

\vspace{\parskip}

\begin{theorem_appendix}[The Master Theorem]
Suppose that for $n \in \Z^+$.
\begin{equation*}
    T(n) =
    \begin{cases}
    c & \text{if $n<B$} \\
    a_1 T(\lceil n/b \rceil) + a_2 T(\lfloor n/b \rfloor) + dn^i & \text{if $n\geq B$}
    \end{cases}
\end{equation*}
where $a_1, a_2, B, b \in \N$.

Let $a = a_1+a_2 \geq 1$, $b>1$, and $c,d,i \in \R \cup \{0\}$. Then,
\begin{equation*}
    T(n) \in
    \begin{cases}
    O(n^i \log n) & \text{if $a=b^i$} \\
    O(n^i) & \text{if $a < b^i$} \\
    O(n^{\log_b a}) & \text{if $a > b^i$}
    \end{cases}
\end{equation*}
\end{theorem_appendix}

Linear Recurrences:

A linear recurrence is an equation
$$
f(n) = \underbrace{a_1 f(n-1) + a_2 f(n-2) + \cdots + a_d f(n-d)}_{\text{homogeneous part}}  + \underbrace{g(n)}_{\text{inhomogeneous part}}
$$
along with some boundary conditions.

The procedure for solving linear recurrences are as follows:

\begin{enumerate}
    \item Find the roots of the characteristic equation. Linear recurrences usually have exponential solutions (such as $x^n$). Such solution is called the \textbf{homogeneous solution}.
    $$
    x^n = a_1 x^{n-1} + a_2 x^{n-2} + \cdots + a_{k-1} x + a_k
    $$
    \item Write down the homogeneous solution. Each root generates one term and the homogeneous solution is their sum. A non-repeated root $r$ generates the term $cr^n$, where $c$ is a constant to be determined later. A root with $r$ with multiplicity $k$ generates the terms
    $$
    d_1r^n \quad d_2nr^n \quad d_3n^2r^n \quad \cdots \quad d_kn^{k-1}r^n
    $$
    where $d_1,\cdots,d_k$ are constants to be determined later.
    \item Find a \textbf{particular solution} for the full recurrence including the inhomogeneous part, but without considering the boundary conditions.
    
    If $g(n)$ is a constant or a polynomial, try a polynomial of the same degree, then of one higher degree, then two higher. If $g(n)$ is exponential in the form $g(n) = k^n$, then try $f(n) = ck^n$, then $f(n) = (bn+c)k^n$, then $f(n) = (an^2+bn+c)k^n$, etc.
    
    \item Write the \textbf{general solution}, which is the sum of homogeneous solution and particular solution.
    \item Substitute the boundary condition into the general solution. Each boundary condition gives a linear equation. Solve such system of linear equations for the values of the constants to make the solution consistent with the boundary conditions.
\end{enumerate}
